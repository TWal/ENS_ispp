\documentclass[12pt]{article}

\input{init.tex}


\usepackage[bottom=2cm,left=2cm,right=2cm,top=2cm]{geometry}

\title{Automated methods for optimizing cache on algebraic data structure}

\author{Thibaut Pérami, Théophile Wallez and Luc Chabassier}


\begin{document}

\maketitle


\section{Abstract}

The speed of modern processors is nearly entirely determined by RAM access and
especially the number of cache misses. Lots of widely use data structure are
quite cache-friendly : array, matrices, but others like linked list, tree or
more complex construction with sum and product types are far more difficult to
optimize.

In this paper we try to implement automated code generation of cache optimized
representation of algebraic types depending of the concrete algorithm
implemented. Currently most functional compiler implement algebraic type with
lots of pointer and dispersed pieces of data across the RAM (depending of the
garbage collector implementation). We try to create parametric implementations
of compacted representation of these structures and to tune them automatically
depending on the input algorithm.

\section{Cache and complex structures}

\subsection{Cache Structure}

TODO : blabla à propos des niveaux des caches du temps d'accès des modèles de
caches utilisé ...

\subsection{Model used}

\subsection{DAG structure}

\section{Naive Implementation}

The naive implementation of algebraic type consist in using different memory
block for every node : data is dispersed in RAM and blocks have pointer to
other blocks.



\section{Parametric Implementation}

\subsection{Block allocation}

\subsection{Compact trees}


\section{Possible Analysis on execution traces}

\subsection{execution logs}

\subsection{Statistics}

\section{Experimental Results}



\section{Conclusion}






\end{document}